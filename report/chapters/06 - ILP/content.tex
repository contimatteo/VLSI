\subsection{Background}
    % \colorbox{BurntOrange}{TODO missing ...}
    This section focuses on the floorplan design problem of VLSI circuits by using a Mixed Linear
    Programming approach (MILP). To implement our model we used the IBM library of 
    \textit{ILOG cplex}. More precisly we choosed to implement the model by using the extension of
    \textit{DOcplex} python modeling API, \textit{Decision and Optization cplex}, which is a 
    library composed of two modules:
    \begin{itemize}
        \item Mathematical Programming Modeling for Python using \textit{docplex.mp}
              (\textit{DOcplex.MP}), which is the module we used.
        \item Constraint Programming Modeling for Python using \textit{docplex.cp} (\textit{DOcplex.CP})
    \end{itemize}

% % % % % % % % % % % % % % % % % % % % % % % % % % % % % % % % % % % % % % % % % % % % % % % % % %
\subsection{Notation}
    A new parameter will be considered in the following solution in order to correct define our model:
    \begin{align*}
        M\        &\ =\ \text{A very large positive number}
    \end{align*}

    Next, we can explicit the model's variables:
    \begin{align*}
        makespan\   &\ =\ \text{height of the plate  } i                     \\
        x_i\    &\ =\ \text{horizontal coordinate of bottom left corner of circuit  } i            \\
        y_i\    &\ =\ \text{vertical coordinate of bottom left corner of of circuit  } i              \\
        u_i\    &\ =\ \begin{cases}
                              1 & \text{if circuit } i \text{ is rotated} \\
                              0 & \text{otherwise}
                          \end{cases}                                        \\
        l_{i,j} &\ =\ \begin{cases}
                              1 & \text{if circuit } i \text{ is on the left side of the circuit} j \\
                              0 & \text{otherwise}
                          \end{cases}                                       \\
        r_{i,j} &\ =\ \begin{cases}
                            1 & \text{if circuit } i \text{ is on the right side of the circuit} j \\
                            0 & \text{otherwise}
                        \end{cases}                                       \\
        a_{i,j} &\ =\ \begin{cases}
                            1 & \text{if circuit } i \text{ is above the circuit} j \\
                            0 & \text{otherwise}
                        \end{cases}                                       \\
        b_{i,j} &\ =\ \begin{cases}
                            1 & \text{if circuit } i \text{ is below the circuit} j \\
                            0 & \text{otherwise}
                        \end{cases}
    \end{align*}

    Moreover for what concern the rotation model, we decide to implement a similar procedure to the
    other previously seen approaches. We implemented a couple of function which have the scope of 
    manage the rotation and the real dimension of each circuit $i$ according to the value of its 
    specific variable rotation $u_i$:
    \begin{subequations}
        \label{ilp:f_dims}
        \begin{align}
            f_h(u_i) &\ = w_i \ u_i + h_i \ (1 - u_i)\\
            f_w(u_i) &\ = h_i \ u_i + w_i \ (1 - u_i)
        \end{align}    
    \end{subequations}
    Please notice that the use of such function will not remove the linearity of each constraints,
    but it is just a name we gave to increase the readability of the report. 
% % % % % % % % % % % % % % % % % % % % % % % % % % % % % % % % % % % % % % % % % % % % % % % % % %
\subsection{Model Formalisation}
    In the following we are going to present the, model formalisation for the VLSI problem as a 
    MILP, by considering the notation previously cited. Let's start with the base model without
    the rotations.
\begin{subequations}
    \label{ilp:base}
    \begin{align}
        \label{ilp:base_obj} \text{minimize}\ makespan                                   &\  &\                       \\
        \label{ilp:base_ycons}  y_c + h_c            &\ \leq\ makespan                       &\ \forall c \in C       \\
        \label{ilp:base_diffn1} x_{c_1} - x_{c_2}    &\ \geq\ w_{c_2} - M\ (1 - l_{c_1c_2})  &\ \forall c_1, c_2 \in C\\ 
        \label{ilp:base_diffn2} x_{c_1} - x_{c_2}    &\ \geq\ w_{c_2} - M\ (1 - r_{c_1c_2})  &\ \forall c_1, c_2 \in C\\
        \label{ilp:base_diffn3} y_{c_1} - y_{c_2}    &\ \geq\ h_{c_2} - M\ (1 - a_{c_1c_2})  &\ \forall c_1, c_2 \in C\\ 
        \label{ilp:base_diffn4} y_{c_1} - y_{c_2}    &\ \geq\ h_{c_2} - M\ (1 - b_{c_1c_2})  &\ \forall c_1, c_2 \in C\\
        \label{ilp:base_diffn5} l_{c_1c_2} + r_{c_1c_2} + a_{c_1c_2} + b_{c_1c_2}  &\ \geq 1 &\ \forall c_1, c_2 \in C\\
        \label{ilp:base_diffn6} l_{c_1c_2} + r_{c_1c_2} + a_{c_1c_2} + b_{c_1c_2}  &\ \leq 2 &\ \forall c_1, c_2 \in C\\
        \label{ilp:base_b_x} 0                       &\ \leq\ x_c \leq\ width - w_c          &\ \forall c \in C       \\
        \label{ilp:base_b_y} 0                       &\ \leq\ y_c \leq\ max\_makespan-h_c    &\ \forall c \in C       \\
        \label{ilp:base_b_makspan} min\_makespan     &\ \leq\ makespan \leq\ max\_makespan   &\ \forall c \in C       \\
        \label{ilp:base_b_a} 0\                      &\ \leq\ a_{c_1c_2}\ \leq\ 1            &\ \forall c_1,c_2 \in C \\
        \label{ilp:base_b_b} 0\                      &\ \leq\ b_{c_1c_2}\ \leq\ 1            &\ \forall c_1,c_2 \in C \\
        \label{ilp:base_b_r} 0\                      &\ \leq\ r_{c_1c_2}\ \leq\ 1            &\ \forall c_1,c_2 \in C \\
        \label{ilp:base_b_l} 0\                      &\ \leq\ l_{c_1c_2}\ \leq\ 1            &\ \forall c_1,c_2 \in C
    \end{align}    
\end{subequations}
    
    Also for what concern ILP, the VLSI problem had been defined with the objective of minimizing 
    the chip height. This can be seen in the model in the equation \ref{ilp:base_obj} where it is
    shown the problem direction, namely the minimization of the integer variable $makespan$.
    In order to complete the project task it is necessary to minimize the variable $makespan$ which
    is an integer variable with lower bound $min\_makespan$ and with upper bound $max\_makespan$.
    Please notice also that the  boundaries $min\_makespan$ and $max\_makespan$ had been computed
    in the same way as for the previous approaches (i.e. SAT, CP and SMT).\\

    For what concern the constraints, the first one to analise is the one that constraints
    (equation \ref{ilp:base_ycons}) the variable $y$ to place blocks for which vertical dimension
    doesn't exceed the value associated to the variable $makespan$.\\

    Now the foucus should be posed on the constraints \ref{ilp:base_diffn1}-\ref{ilp:base_diffn6}.
    These constraints achieve the same task as the \textit{diffn} constraint in CP. More in
    details, we implemented this constraint in ILP by imposing a set of disequations. As we know
    it is defined in CP as \ref{eq:diffn}. So, in order to have the same result of the disjoint
    clause conditions, we need to control the variable's results. The constraints 
    \ref{ilp:base_diffn1}-\ref{ilp:base_diffn4} states the position of a circuit $c_1$ respect to
    another circuit $c_2$. The auxiliary parameter $M$ has the scope to falsify the disequation
    when the corrispondent variable is set to 0. Then the subsequent constraints 
    \ref{ilp:base_diffn5}-\ref{ilp:base_diffn6} are needed to allow the sum of the binary variables 
    (namely $r_{c_1c_2}$, $l_{c_1c_2}$, $a_{c_1c_2}$ and $b_{c_1c_2}$) belonging in in the range
    between $1$ and $2$. \\
    
    Finally the last ones are just bounding constraints, which describe the upper and lower bounds
    of the variables. More in details, while the integer variables $x_c$ can be in the range 
    between $0$ and $width-widths_c$ (see constraint \ref{ilp:base_b_x}), the variables $y_c$ can have
    values in the range between $0$ and $makespan-heights_c$ (see constraint \ref{ilp:base_b_y}). 
    This is due to the fact that we cannot place boxes in a position such that the chip's dimension 
    exceed the circuit plate both in high and in width. Then, since the all the variables of the 
    constraints \textit{diffn} are binary, their value should fall in the range $0$ and $1$.
% % % % % % % % % % % % % % % % % % % % % % % % % % % % % % % % % % % % % % % % % % % % % % % % % %

\subsection{Rotation}
    % \colorbox{BurntOrange}{TODO }
    Next step consists to extend the model in order take into account also the box rotation.
    For this scope we introduced, as we did also in the previous sections, a new set of bianary 
    variables $u_c$. 

    \begin{subequations}
        \label{ilp:rot}
        \begin{align}
        \label{ilp:rot_obj} \text{minimize}\ makespan                                   &\  &\                       \\
        \label{ilp:rot_ycons}  y_c + f_h(u_c)       &\ \leq\ makespan                       &\ \forall c \in C       \\
        \label{ilp:rot_diffn1} x_{c_1} - x_{c_2} &\ \geq\ f_w(u_{c_2}) - M\ (1-l_{c_1c_2})  &\ \forall c_1, c_2 \in C\\ 
        \label{ilp:rot_diffn2} x_{c_1} - x_{c_2} &\ \geq\ f_w(u_{c_2}) - M\ (1-r_{c_1c_2})  &\ \forall c_1, c_2 \in C\\ 
        \label{ilp:rot_diffn3} y_{c_1} - y_{c_2} &\ \geq\ f_h(u_{c_2}) - M\ (1-a_{c_1c_2})  &\ \forall c_1, c_2 \in C\\ 
        \label{ilp:rot_diffn4} y_{c_1} - y_{c_2} &\ \geq\ f_h(u_{c_2}) - M\ (1-b_{c_1c_2})  &\ \forall c_1, c_2 \in C\\ 
        \label{ilp:rot_diffn5} l_{c_1c_2} + r_{c_1c_2} + a_{c_1c_2} + b_{c_1c_2}  &\ \geq 1 &\ \forall c_1, c_2 \in C\\
        \label{ilp:rot_diffn6} l_{c_1c_2} + r_{c_1c_2} + a_{c_1c_2} + b_{c_1c_2}  &\ \leq 2 &\ \forall c_1, c_2 \in C\\
        \label{ilp:rot_b_x} 0                       &\ \leq\ x_c \leq\ width - w_c          &\ \forall c \in C       \\
        \label{ilp:rot_b_y} 0                    &\ \leq\ y_c \leq\ max\_makespan-f_h(u_c)  &\ \forall c \in C       \\
        \label{ilp:rot_b_makspan} min\_makespan     &\ \leq\ makespan \leq\ max\_makespan   &\ \forall c \in C       \\
        \label{ilp:rot_b_a} 0\                      &\ \leq\ a_{c_1c_2}\ \leq\ 1            &\ \forall c_1,c_2 \in C \\
        \label{ilp:rot_b_b} 0\                      &\ \leq\ b_{c_1c_2}\ \leq\ 1            &\ \forall c_1,c_2 \in C \\
        \label{ilp:rot_b_r} 0\                      &\ \leq\ r_{c_1c_2}\ \leq\ 1            &\ \forall c_1,c_2 \in C \\
        \label{ilp:rot_b_l} 0\                      &\ \leq\ l_{c_1c_2}\ \leq\ 1            &\ \forall c_1,c_2 \in C \\
        \label{ilp:rot_b_u} 0\                      &\ \leq\ u_{c_1c_2}\ \leq\ 1            &\ \forall c_1,c_2 \in C
        \end{align}    
    \end{subequations}

    As we previously said, the use of the functions defined in the equation \ref{ilp:f_dims}, are 
    needed to express the correct dimension to the model to each box in each condition. In this way 
    it is possible to extend the model in a way that it can be possible to preserve the 
    formalisation of the no rotations model, i.e. the base one (see equations \ref{ilp:base}).
    Indeed, looking at the model, we didn't any changes to the objective function. We just modified
    the constraints \ref{ilp:rot_diffn1}-\ref{ilp:rot_diffn4}, so that they take into account the 
    correct dimensions and similarly the constraints \ref{ilp:rot_ycons} and \ref{ilp:rot_b_y}.
% % % % % % % % % % % % % % % % % % % % % % % % % % % % % % % % % % % % % % % % % % % % % % % % % %

% \subsection{Search}
%     \colorbox{BurntOrange}{TODO missing ...}

% % % % % % % % % % % % % % % % % % % % % % % % % % % % % % % % % % % % % % % % % % % % % % % % % %

\subsection{Results}
    \colorbox{BurntOrange}{TODO missing ...}
