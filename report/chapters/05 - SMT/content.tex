
\subsection{Background}
    \textit{Satisfiability Modulo Theories (SMT)} is the problem of determining whether a
    mathematical formula is satisfiable. It generalizes the Boolean satisfiability problem (SAT) to
    more complex formulas involving real numbers, integers, and/or various data structures such as 
    lists, arrays, bit vectors, and strings.

% % % % % % % % % % % % % % % % % % % % % % % % % % % % % % % % % % % % % % % % % % % % % % % % % %

\subsection{Formulation}
    \colorbox{BurntOrange}{TODO missing ...} \\

    In order to write a proper SMT formulation, we started from the SAT formulation (available in
    the Section \ref*{chapter:SAT}) and we have replaced the boolean representation of integer 
    operators with these operators themself.\\

    In particular, the following replacements have been done:
    \begin{align*}
      \text{SMT} \ &\ \leftarrow\ \text{SAT}      \\
                 \cline{1-2}
        a \geq 1 \ &\ \leftarrow\ \text{at\_least\_one}(a) \\
           a < b \ &\ \leftarrow\ lt(a,b)         \\
        a \leq b \ &\ \leftarrow\ lte(a,b)        \\
           a > b \ &\ \leftarrow\ gt(a,b)         \\
        a \geq b \ &\ \leftarrow\ gte(a,b)   
    \end{align*}

    The final formulation obtained is very "near" to the one defined for CP in the
    Section \ref{chapter:CP}.

% % % % % % % % % % % % % % % % % % % % % % % % % % % % % % % % % % % % % % % % % % % % % % % % % %

\subsection{Results}
    \colorbox{BurntOrange}{TODO missing ...} \\
