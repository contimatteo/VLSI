\subsection{Background}\label{subsec:background}
    
SAT problems are based only on boolean variables. We have to represent positive integer numbers. 
One-hot encoding is a possible solution for representing integers using only boolean variables.\\
How does one-hot encoding works:\\
For every integer variable we declare one boolean variable for every item of the domain of 
the numeric variable. Only one boolean per integer variable is set to \texttt{True}. The only \texttt{True} 
variable corresponds to the value of the domain that is being represented.\\
This solution leads to a simple implementation of the operations but is not very efficient 
because it requires the SAT solver to manage a very big quantity of variables and clauses.

We decided to use decimal encoding. Let $domain\_max$ be the maximum value of the domain of 
a positive integer variable. The number of boolean variables allocated to represent it is:
\begin{equation}
    var\_size = \lceil \log_2 (domain\_max + 1)\rceil
\end{equation}
We have a list of boolean variables that take the value of the corresponding bit of the binary encoding
of the positive integer. We implemented specific functions for handling mathematical operations.
They are discussed in \ref*{subsec:predicates}
\\\\
Implementation of the models:
we decided to create a default abstract model that has the common parts that will be needed by
every concrete model. The abstract model is extended by the Base model and the Rotation model. 
The rotation model is discussed in its section \ref*{subsec:rotation}

% % % % % % % % % % % % % % % % % % % % % % % % % % % % % % % % % % % % % % % % % % % % % % % % % %

\subsection{Variables}\label{subsec:variables}
The variables for the SAT solver are lists of boolean variables but their meaning does not differ from the 
definitions in \ref{sec:shared_variables}.\\
Several variables are used only locally in functions and they are defined there.\\
Being numbers represented with lists of boolean the notation for the $i$-th element of list 
$list$ is $list_i$.

% % % % % % % % % % % % % % % % % % % % % % % % % % % % % % % % % % % % % % % % % % % % % % % % % %

\subsection{Predicates and Functions}\label{subsec:predicates}
We created several support functions for dealing with the conversion from integers to lists of
boolean variables and back.\\
If an integer value is provided then it is converted to a list of python \texttt{bool} variables
that are consistent with 
\texttt{z3 Bool} variables and can be used to do calculations.\\
In this way we achieve the transparency of the mathematical functions for the type of operand 
used.\\
The length of the lists of booleans is calculated upon the maximum value of the domain of the
variable. Padding of the shortest list is performed if necessary in case of operations between
 two values of different domain sizes.
\\\\
Unary logic operation functions:\\
\texttt{\textbf{all\_F(list):}} 
\begin{equation}
    \bigwedge\limits_{b \in list}\neg b
\end{equation}

\texttt{\textbf{at\_least\_one(list):}}
\begin{equation}
        \bigvee\limits_{b \in list} b
\end{equation}

\texttt{\textbf{at\_most\_one(list):}}\\
We implemented the Heule encoding making this a recursive function.\\
The base case of the recursion is if the list has less then four elements. In this case the 
pair-wise encoding is used to calculate the result.\\
Let n be the length of $list$
\begin{equation}
        result\_base = \bigwedge\limits_{0 < i < n} \ \ \bigwedge\limits_{i+1 \leq j \leq n}\neg (b_i \wedge b_j)
\end{equation}
In the recursive case we split $list$ into two parts: $list_A$ has length two, $list_B$ is
 the remainder. We introduce a new variable $y$. We append $y$ to $list_A$ and ¬$y$ to 
 $list_B$. We then calculate the result through recursion:
\begin{equation}
    result\_recursive = at\_most\_one(list_A) \wedge at\_most\_one(list_B)
\end{equation}

This encoding achieves a number of clauses of $3n - 6$ using $(n-3)/2$ new variables.
\\

\texttt{\textbf{exactly\_one(list):}}
\begin{equation}
    at\_most\_one(list) \wedge at\_least\_one(list)
\end{equation}


\texttt{\textbf{ne(l1, l2):}} Not equal\\
We must consider the cases where the two lists have different lengths.
Let $min\_len$ be the minimum between the length of the lists, $exc_1$ be the part of $l1$
 exceeding $min\_len$ and $exc_2$ be the part of $l2$ exceeding $min_len$. At least one 
among $exc_1$ and $exc_2$ will be an empty list and will provide no clauses.\\
The result is computed as follows:
\begin{equation}
    at\_least\_one(exc_1) \vee at\_least\_one(exc_2) \vee \underline{\bigvee}\limits_{i \leq min\_len}(l1_i, l2_i)
    %%% any other way to represent the XOR?
\end{equation}

\texttt{\textbf{eq(l1, l2):}} Equal\\
One of the lists is eventually padded with bits set to \texttt{False} then we can
calculate the result:
\begin{equation}
    \bigwedge\limits_{0 \leq i < n} \neg (l1[i]      \ \ \underline\vee\ \ l2[i] )
\end{equation}


\texttt{\textbf{gte(l1, l2):}} Greater then or equal\\
We must consider the cases where the two lists have different lengths.
Let $min\_len$ be the minimum between the length of the lists, $exc_1$ be the part of $l1$ 
exceeding $min\_len$ and $exc_2$ be the part of $l2$ exceeding $min\_len$. At least one among
$exc_1$ and $exc_2$ will be an empty list and will provide no clauses.\\
The result is computed as follows:
\begin{equation}
    at\_least\_one(exc_1) \vee ( all\_F(exc_2) \wedge gte\_same\_len(l1\_cut, l2\_cut))
\end{equation}
Where $l1\_cut$ and $l2\_cut$ are the result of pruning $l1$ and $l2$ at length $min\_len$.\\

\texttt{\textbf{gte\_same\_len(l1, l2):}} Greater then or equal of two list of the same length\\
The encoding used is the AND encoding with Common Subexpression Elimination as described in 
% \cite{Elgabou} and \cite{Zhao}\\

\texttt{\textbf{gt(l1, l2):}} Greater then\\
We must consider the cases where the two lists have different lengths.
Let $min\_len$ be the minimum between the length of the lists, $exc_1$ be the part of $l1$ 
exceeding $min\_len$ and $exc_2$ 
be the part of $l2$ exceeding $min\_len$. At least one among $exc_1$ and $exc_2$ will be an
empty list and will provide no clauses.\\
The result is computed as follows:
\begin{equation}
    at\_least\_one(exc_1) \vee ( all\_F(exc_2) \wedge gt\_same\_len(l1\_cut, l2\_cut))
\end{equation}

\texttt{\textbf{gt\_same\_len(l1, l2):}} Greater then of two list of the same length\\
The encoding used is the AND encoding with Common Subexpression Elimination. We slightly
% modified the implementation of \cite{Zhao} to exclude the equality.\\
Len $n$ be the length of the lists $l1$, $l2$.\\
If $n=1$:
\begin{equation*}
    l1_0 \wedge \neg l2_0
\end{equation*}\\

Otherwise:\\
Let $x$ be a list of boolean variables of length $n-1$
\begin{equation*}
    first =  l1_0 \wedge \neg l2_0 %first
\end{equation*}
%
\begin{equation*}
    second = x_0 \Longleftrightarrow \neg (l1_0 \underline\vee l2_0) %second
\end{equation*}
%
\begin{equation*}
    third = \bigwedge\limits_{0 \leq <n-2} x_{i+1} \Longleftrightarrow x_i \wedge (\not (l1_{i+1} \ \ \underline\vee \ \ l2_{i+1}))%third
\end{equation*}
%
\begin{equation*}
    fourth = \bigvee\limits_{0 \leq i < n-1} x_i \wedge (l1_{i+1} \wedge \neg l2_{i+1})
\end{equation*}
\begin{equation*}
    result = first \vee (second \wedge third \wedge fourth)
\end{equation*}

\texttt{\textbf{lt(l1, l2):}} Less then\\
\begin{equation}
    result = gt(l2, l1)
\end{equation}
Notice the reverse order of the parameters\\

\texttt{\textbf{lte(l1, l2):}} Less then of two lists of the same length\\
\begin{equation}
    result = gte(l2, l1)
\end{equation}
Notice the reverse order of the parameters\\

\texttt{\textbf{sum\_b(l1, l2):}} sum of two numbers\\
Padding is performed on the shortest list if need be.
Let $n$ be the length of the lists, eventually padded. Let $carry$ be a list of boolean
variables of length $n$. $carry_0$ is set to \texttt{false}.  
\begin{equation}
    carry_{i+1} = ((l1_i \underline\vee l2_i) \wedge carry_i) \vee (l1_i \wedge l2_i)  \  \ \forall i \in [0, n]
\end{equation}
\begin{equation}
        result_i = (l1_i \ \ \underline\vee \ \ l2_i) \ \ \underline\vee \ \ carry_i \  \ \forall i \in [0, n-1]
\end{equation}

$result$ is a list that contains the sum of the numbers. $carry_n$ could be used to
notify an overflow but we did  not use it because the domains size are designed to avoid this.\\

\texttt{\textbf{sub\_b(l1, l2):}} subtraction of two numbers\\
Padding is performed on the shortest list if need be.
Let $n$ be the length of the lists, eventually padded.
Let $borr$ be a list a list of boolean variables of length $n$. $borr_0$ is set to \texttt{false} 
\begin{equation}
    result_i = (l1_i \ \ \underline\vee \ \ l2_i) \ \ \underline\vee \ \ borr_i \  \ \forall i \in [0, n-1]
\end{equation}
\begin{equation}
    borr_{i+1} = ((\neg (l1_i \ \ \underline\vee \ \ l2_i) \wedge borr_i) \vee (\neg l1_i \wedge \ \ l2_i)) \  \ 
    \forall i \in [0, n-1]
\end{equation}

disjunctive = used for cumulative:
disjunctive logic formula


fzn\_cumulative:
logic description

cumulative:
check if disjunctive can be applied:
OR(gt(r[c]+ min\_r, boundary), c == idx\_min\_r) forall c in C

disjunctive => disxunctive(x, dx)
not disjunctive => \_fzn\_cumulative(...)

symmetrical: calculates the symmetrical position of a coordinate wrt the boundary
Xsymm = end - (x - start + dx[i])

axial symmetry:
x\_symm = symmetrical()
put that into logic or math





\subsection{Constraints}

\textbf{Static constraints}\\
The process of finding the best solution is iterative. Some constraints do not 
need to be modified along the iteration process because the values on which they 
are based do not change during the iteration. They are constraints on the horizontal 
direction of the plate.\\

\texttt\textbf{diffn(x, y, heights, widths)}\\
The diffn constraint, DISCUSSED IN SECTION X is implemented as follow.
\begin{equation}
    \begin{split}
        \bigwedge\limits_{(i, j) \in CC}&=
        lte(sum\_b(x_i, widths_i), x_j) \vee \\
        &lte(sum\_b(y_i, heights_i), y_j) \vee \\
        &lte(sum\_b(x_j, widths_j), x_i) \vee \\
        &lte(sum\_b(y_j, heights_j), y_i)
    \end{split}
\end{equation}\\

\texttt\textbf{NAME} Circuits must not be placed over the side of the board\\
\begin{equation}
    \bigwedge\limits_{c \in C} lte(x_c, sub_b(width, widths_c))
\end{equation}\\

\texttt\textbf{cumulative(y, heights, widths, width, minw, id\_x) }\\ 
Optionally we can impose a cumulative constraint along the x axis. This is an implied
 constraint. FORMUA LOGICA?\\

\textbf{Dynamic constraint}
Dynamic constraints are the constraint whose values change during the iteration so we
need to update their limits when we perform the optimization. They are the constraints 
on the height of the board. We use z3.push() at the beginning of every loop and z3.pop()
 at the end to avoid jamming the solver with out of date constraints.\\

\texttt\textbf{NAME} Circuits must not be placed over the current limit of height: 
$target\_makespan$ \\
\begin{equation}
    \bigwedge\limits_{c \in C} lte(y_c, sub\_b(makespan, heights_c))
\end{equation}\\

\texttt\textbf{cumulative(x,widths,heights, makespan, min\_h, idx)} \\
Optionally we can impose a cumulative constraint along the y axis. This is an implied
constraint. FORMUA LOGICA?

symmetry braking constraints: 
static: 
     we tried also biggest circuit in 0,0 but?
    axial symmetry(w) FORMUA LOGICA
dynamic: (axial symmetry) FORMUA LOGICA


% % % % % % % % % % % % % % % % % % % % % % % % % % % % % % % % % % % % % % % % % % % % % % % % % %

\subsection{Rotation}\label{subsec:rotation}
    \colorbox{BurntOrange}{TODO missing ...}

% % % % % % % % % % % % % % % % % % % % % % % % % % % % % % % % % % % % % % % % % % % % % % % % % %

\subsection{Search}\label{subsec:search}
    \colorbox{BurntOrange}{TODO missing ...}

% % % % % % % % % % % % % % % % % % % % % % % % % % % % % % % % % % % % % % % % % % % % % % % % % %

\subsection{Results}\label{subsec:results}
    \colorbox{BurntOrange}{TODO missing ...}
