
\subsection{Background}
The first paradigm used to solve the VLSI optimization problem is Constraint Programming (CP),
developed using MiniZinc as modeling language.
Starting from a base model following the constraints discussed in Section \ref{sec:shared_constraints},
we added: rotation variables $r_i$, symmetry breaking constraints and search strategies. We have a model
for each combination of the previous elements.
To test different behaviour of solvers we chose to run each model on both Chuffed and Gecode solvers.

% % % % % % % % % % % % % % % % % % % % % % % % % % % % % % % % % % % % % % % % % % % % % % % % % %

\subsection{Notation} \label{sec:CP_notation}
In CP we tried to break also the symmetries related to \textit{virtual} circuits, but we decided to
keep only the simplest ones for efficiency reasons, as anticipated in Section \ref{sec:symmetries}.

This section is mainly needed to introduce constants related to \textit{virtual} circuits:

\begin{align*}
  vc_{i,j} & \ =\ \text{\textit{virtual} circuit obtained from  } (i,j) \in CC      \\
  n\_vc\   & \ =\ \text{maximum number of \textit{virtual} circuits}                \\
           & \ =\ |CC| = \frac{nc!}{2! \cdot (nc - 2)!} = \frac{nc \cdot (nc-1)}{2} \\
  n\_rvc\  & \ =\ n\_vc + nc                                                        \\
  VC       & \ =\ \{ c_i\ |\ i \in [1,n\_vc] \}                                     \\
  VV       & \ =\ \{ (i,j)\ \in VC \times VC\ |\ i <j \}                            \\
  RVC      & \ =\ \{ c_i\ |\ i \in [1,n\_rvc] \}                                    \\
  % c_pairs &\ =\ [(i,j)\ |\ (i,j) \in CC]
\end{align*}

The first definition needs deeper explanation: a rectangle can be defined giving the position
of its bottom left and its top right corner. For a \textit{virtual} circuit the corners are the
bottom left corner of circuit $i$ and the top right corner of circuit $j$.
The maximum number of virtual circuits is then given by the number of possible combinations
(without repetition) of the circuit indexes grouped in pairs, which is also the cardinality of $CC$.



% % % % % % % % % % % % % % % % % % % % % % % % % % % % % % % % % % % % % % % % % % % % % % % % % %

\subsection{Functions \& Predicates} \label{sec:CP_functions_predicates}
The additional functions and predicates we define in this section will later be used in the models
that adopt symmetry breaking constraints or that allow the rotation of circuits. The other models
work with the constraints already defined in Section \ref{sec:shared_constraints}.

\paragraph{Functions}
\begin{align}
  vc\_x(vc_{i,j})\        =\  & min(x_i, x_j)                                                     \nonumber \\
  vc\_y(vc_{i,j})\        =\  & min(y_i, y_j)                                                     \nonumber \\
  vc\_width(vc_{i,j})\    =\  & max(x_i + w_i, x_j + w_j) - vc\_x(vc_{i,j})                       \nonumber \\
  vc\_height(vc_{i,j})\   =\  & max(y_i + h_i, y_j + h_j) - vc\_y(vc_{i,j})                       \nonumber \\
  r\_w(c) =\                  & bool2int(\neg\ is\_rotated_c) \cdot w_c + bool2int(is\_rotated_c) \cdot h_c
  \label{eq:CP_r_w}                                                                                         \\
  r\_h(c) =\                  & bool2int(\neg\ is\_rotated_c) \cdot h_c + bool2int(is\_rotated_c) \cdot w_c
  \label{eq:CP_r_h}
\end{align}

where $vc\_x(vc_{i,j})$, $vc\_y(vc_{i,j})$, $vc\_width(vc_{i,j})$ and $vc\_height(vc_{i,j})$
return respectively the $x_{vc_{i,j}}$, $y_{vc_{i,j}}$, $w_{vc_{i,j}}$ and $h_{vc_{i,j}}$ of
the \textit{virtual} circuit $vc_{i,j}$, while $r\_w(c)$ and $r\_h(c)$ check if circuit $c$
is rotated and update coherently $w_c$ and $h_c$.

\paragraph{Predicates}
% \begin{align*}
%     vc\_is\_valid(vc_{i,j})     &\leftarrow\    \bigwedge_{c \in C} &  (\neg(x_c < vc_x \land x_c + w_c > vc_x) \land
%                                                 \neg(x_c + w_c > vc_x + vc_w))                                  \\
%                                             &&  \lor                                             \\
%                                             &&  (\neg(y_c < vc_y \land y_c + h_c > vc_y) \land
%                                                 \neg(y_c + h_c > vc_y + vc_h))                                  \\
%     c\_equal\_dim\_symmetry &\leftarrow\        \bigwedge_{(c_1, c_2) \in CC} & (w_{c_1} == w_{c_2} \land h_{c_1} == h_{c_2})                   \\
%                                             &&  \rightarrow lex\_lesseq([x_{c_1}, y_{c_1}], [x_{c_2}, y_{c_2}]) \\ 
%     vc\_equal\_dim\_symmetry &\leftarrow\       \bigwedge_{(c, vc) \in C \times VC} & (vc\_is\_valid(vc) \land w_{c_1} == w_{c_2} \land h_{c_1} == h_{c_2})   \\
%                                             &&  \rightarrow lex\_lesseq([x_{c_1}, y_{c_1}], [x_{c_2}, y_{c_2}])     \\      
%     vv\_equal\_dim\_symmetry &\leftarrow\       \bigwedge_{(vc_1, vc_2) \in VV}
%                                             &   (vc\_is\_valid(vc_1) \land vc\_is\_valid(vc_2) \land                         \\
%                                             &&  \neg (vc_{1x} == vc_{2x}) \land \\
%                                             &&  vc_{1w} == vc_{2w} \land vc_{1h} == vc_{2h} \land \\
%                                             &&  (|vc_{1x} - vc_{2x}| \geq vc_{1w} \lor |vc_{1y} - vc_{2y}| \geq vc_{1h})) \\
%                                             &&  \rightarrow lex\_lesseq([vc_{1x}, vc_{1y}], [vc_{2x}, vc_{2y}]) \\
%     c\_consecutive\_on\_x(c_1, c_2) &\leftarrow  &   x_{c_1} + w_{c_1} == x_{c_2} \lor x_{c_2} + w_{c_2} == x_{c_1} \\
%     c\_consecutive\_on\_y(c_1, c_2) &\leftarrow  &   y_{c_1} + h_{c_1} == y_{c_2} \lor y_{c_2} + h_{c_2} == y_{c_1} \\
%     c\_can\_be\_swapped\_on\_x(c_1, c_2) &\leftarrow & c\_consecutive\_on\_x(c_1, c_2) \land y_{c_1} == y_{c_2} \land h_{c_1} == h_{c_2} \\
%     c\_can\_be\_swapped\_on\_y(c_1, c_2) &\leftarrow & c\_consecutive\_on\_y(c_1, c_2) \land x_{c_1} == x_{c_2} \land w_{c_1} == w_{c_2} \\ 
%     c\_consecutive\_symmetry &\leftarrow  \bigwedge_{(c_1, c_2) \in CC} & (c\_can\_be\_swapped\_on\_x(c_1, c_2) \rightarrow lex\_less([ x_{c_1} ], [ x_{c_2} ])) \land \\
%                                             &&  (c\_can\_be\_swapped\_on\_y(c_1, c_2) \rightarrow lex\_less([ y_{c_1} ], [ y_{c_2} ]))     
% \end{align*}
\begin{align}
  \text{vc\_valid}(vc_{i,j})      \leftarrow & \ \hspace{0.45cm} \bigwedge_{c \in C} \hspace{0.3cm} (\neg(x_c < vc_x \land x_c + w_c > vc_x) \land \neg(x_c + w_c > vc_x + vc_w))\ \lor      \nonumber           \\
                                             & \ \hspace{1.3cm} \lor (\neg(y_c < vc_y \land y_c + h_c > vc_y) \land \neg(y_c + h_c > vc_y + vc_h))                                                               \\
                                             & \nonumber                                                                                                                                                         \\
  \text{c\_eq\_dim\_sym}          \leftarrow & \ \bigwedge_{(c_1, c_2) \in CC} (w_{c_1} == w_{c_2} \land h_{c_1} == h_{c_2}) \rightarrow lex\_lesseq([x_{c_1}, y_{c_1}], [x_{c_2}, y_{c_2}])                     \\
                                             & \nonumber                                                                                                                                                         \\
  \text{vc\_eq\_dim\_sym}         \leftarrow & \ \bigwedge_{(c, vc) \in C \times VC} \hspace{0.3cm} (\text{vc\_valid}(vc) \land w_{c} == w_{vc} \land h_{c} == h_{vc}) \rightarrow               \nonumber       \\
                                             & \hspace{2.2cm} \rightarrow lex\_lesseq([x_{c}, y_{c}], [x_{vc}, y_{vc}])                                                                                          \\
                                             & \nonumber                                                                                                                                                         \\
  \text{vv\_eq\_dim\_sym}         \leftarrow & \ \bigwedge_{(vc_1, vc_2) \in VV} (\text{vc\_valid}(vc_1) \land \text{vc\_valid}(vc_2) \land \neg (vc_{1x} == vc_{2x})\ \land                           \nonumber \\
                                             & \hspace{2cm} \land vc_{1w} == vc_{2w} \land vc_{1h} == vc_{2h} \land \nonumber                                                                                    \\
                                             & \hspace{2cm} \land (|vc_{1x} - vc_{2x}| \geq vc_{1w} \lor  |vc_{1y} - vc_{2y}| \geq vc_{1h}))  \rightarrow              \nonumber                                 \\
                                             & \hspace{2cm} \rightarrow lex\_lesseq([vc_{1x}, vc_{1y}], [vc_{2x}, vc_{2y}])                                                                                      \\
                                             & \nonumber                                                                                                                                                         \\
  \text{c\_x\_consec}(c_1, c_2)   \leftarrow & \hspace{0.5cm} x_{c_1} + w_{c_1} == x_{c_2} \lor x_{c_2} + w_{c_2} == x_{c_1}                                                                                     \\
  \text{c\_y\_consec}(c_1, c_2)   \leftarrow & \hspace{0.5cm} y_{c_1} + h_{c_1} == y_{c_2} \lor y_{c_2} + h_{c_2} == y_{c_1}                                                                                     \\
  \text{c\_on\_x\_swap}(c_1, c_2) \leftarrow & \hspace{0.5cm} \text{c\_x\_consec}(c_1, c_2) \land y_{c_1} == y_{c_2} \land h_{c_1} == h_{c_2}                                                                    \\
  \text{c\_on\_y\_swap}(c_1, c_2) \leftarrow & \hspace{0.5cm} \text{c\_y\_consec}(c_1, c_2) \land x_{c_1} == x_{c_2} \land w_{c_1} == w_{c_2}                                                                    \\
                                             & \nonumber                                                                                                                                                         \\
  \text{c\_consec\_sym}           \leftarrow & \ \bigwedge_{(c_1, c_2) \in CC} (\text{c\_on\_x\_swap}(c_1, c_2) \rightarrow lex\_less([ x_{c_1} ], [ x_{c_2} ]))\ \land                                \nonumber \\
                                             & \ \hspace{1.6cm} \land (\text{c\_on\_y\_swap}(c_1, c_2) \rightarrow lex\_less([ y_{c_1} ], [ y_{c_2} ]))
\end{align}

\hfill \\
As mentioned in Section \ref{sec:CP_notation}, $vc_{i,j}$ defines a rectangle in the plate, but, in order to
be defined as \textit{virtual} circuit, its edges must not cross any \textit{real} circuit.
This condition is checked by the predicate $vc\_valid(vc_{i,j})$, where
$vc_x = vc\_x(vc_{i,j})$, $vc_y = vc\_y(vc_{i,j})$, $vc_w = vc\_width(vc_{i,j})$, $vc_h = vc\_height(vc_{i,j})$.
The predicates $c\_eq\_dim\_sym$, $vc\_eq\_dim\_sym$, $vv\_eq\_dim\_sym$ apply
lexicographic order to circuits with same dimensionality; in particular the first predicate keeps in consideration only
\textit{real} circuits, the second a \textit{real} circuit and a \textit{virtual} circuit, the third only couples of
\textit{virtual} circuits. The last one must also check that the circuits $c_1$ and $c_2$ do not overlap,
otherwise it would be in contrast with any lexicographic order constraint, making the solution unfeasible.
Another possible case in which a couple of circuit can be swapped is when they have a shared side with same length;
the last predicates are needed to catch those situations and apply to those couple of circuits the lexicographic order.

% % % % % % % % % % % % % % % % % % % % % % % % % % % % % % % % % % % % % % % % % % % % % % % % % %

\subsection{Constraints}
\subsubsection{Base} \label{sec:CP_base}
The reference model is the one with the constraints described in Section \ref{sec:shared_constraints}
with the following simple symmetry breaking constraint:
\begin{equation*}
  x_1 <= x_2 \land y_1 <= y_2
\end{equation*}

which is a simpler version of:

\begin{equation*}
  lex\_lesseq(x\_v, x\_v') \land lex\_lesseq(y\_v, y\_v')
\end{equation*}

where $x\_v$, $x\_v'$, $y\_v$, $y\_v'$ are defined in \ref{eq:specular_coord}.

% % % % % % % % % % % % % % % % % % % % % % % % % % % % % % % % % % % % % % % % % % % % % % % % % %

\subsubsection{Rotation} \label{sec:CP_rotation}

In order to introduce rotations to the model described in Section \ref{sec:CP_base} or any
of the following CP models, we need to add rotation variables $r_i$, functions \ref{eq:CP_r_w}, \ref{eq:CP_r_h}. We
also need to modify already existing constraints substituting $w_c$ with $r\_w(c)$ and $h_c$ with $r\_h(c)$.

\subsubsection{Symmetry} \label{sec:CP_symmetry}

As already discussed in Section \ref{sec:symmetries}, we decided to break only a few symmetries among all the ones we
detected. The following are the constraints added to the models in order to deal with symmetries.\\

First of all we removed the solutions specular w.r.t. the horizontal and vertical axis [Fig.\ref{fig:symmetry_specular}]:

\begin{equation*}
  lex\_lesseq(x\_v, x\_v') \land lex\_lesseq(y\_v, y\_v')
\end{equation*}
with $x\_v$, $x\_v'$, $y\_v$, $y\_v'$ already defined in \ref{eq:specular_coord}.

Then we avoid swapping of circuits with same dimensions [Fig.\ref{fig:symmetry_swap}, \ref{fig:vc_swap}],
which obviously lead to a different solution, but with same \textit{makespan}:
\begin{equation*}
  \text{c\_eq\_dim\_sym} \land \text{vc\_eq\_dim\_sym} \land \text{vv\_eq\_dim\_sym}
\end{equation*}
As demonstration of the inefficiency of \textit{virtual} circuit constraint, we will show 
separately in the results [Section \ref{sec:CP_results}] a model with only \(c\_eq\_dim\_sym\)
and a model with also \(vc\_eq\_dim\_sym\ \land vv\_eq\_dim\_sym\) (the last one will have "virtual" in the name).\\

The case we simplified is the one figured in [Fig.\ref{fig:vc_specular_in}]: instead of applying
lexicographic order to all \textit{real} circuits within a generic \textit{virtual} circuit, we
decided to limit ourself to \textit{virtual} circuits containing only two \textit{real} circuits.
We can see this like avoiding adjacent circuits, having the shared edge with same length,
to swap their position:
\begin{equation*}
  \text{c\_consec\_sym} \leftrightarrow \top
\end{equation*}
in order to have better understanding of the predicates above, recover Section \ref{sec:CP_functions_predicates}.

% % % % % % % % % % % % % % % % % % % % % % % % % % % % % % % % % % % % % % % % % % % % % % % % % %

\subsection{Search} \label{sec:CP_search}
The performances of all the models are compared with and without the search strategy we selected.
Actually in the input of all models the circuits are sorted in decreasing order according to their area,
but then we also implemented a sequential search
\footnote[2]{https://www.minizinc.org/doc-2.5.5/en/mzn\_search.html\#search-annotations}
with the following search annotations, in this specific order:
\begin{enumerate}
  \item ann\_search\_makespan = int\_search([ $makespan$ ], input\_order, indomain\_split)
  \item ann\_search\_x = int\_search($x$, input\_order, indomain\_min)
  \item ann\_search\_y = int\_search($y$, input\_order, indomain\_min)
\end{enumerate}

Models with both rotation and search strategies at the end of the sequential search
list also have:
\begin{enumerate}[resume]
  \item ann\_search\_rot = bool\_search($[r_c\ |\ c \in C]$, input\_order, indomain\_min)
\end{enumerate}

To avoid getting stuck during the search of the solution we added also luby restart strategy
\footnote[3]{https://www.minizinc.org/doc-2.5.5/en/mzn\_search.html\#restart},
choosing empirically the value of parameter $scale$, where \(scale\) is an integer defining 
after how many nodes to restart. For Luby restart the \textit{k-th} restart gets 
\(scale \cdot L[k]\) where \(L[k]\) is the \textit{k-th} number in the Luby sequence.

% % % % % % % % % % % % % % % % % % % % % % % % % % % % % % % % % % % % % % % % % % % % % % % % % %

\subsection{Results} \label{sec:CP_results}

\paragraph{Hardware specifcations}
All the experiments for the CP technology have been executed on a laptop computer running Linux Mint 20.3, Linux kernel 5.15
equipped with the following hardware:
\texttt{Intel(R) Core(TM) i5-8265U CPU @ 1.60GHz, 16Gb Ram 2.4GHz}.\\

The results obtained for the \(base\) models are plotted in the Figures [\ref{fig:CP_results_base1},\ref{fig:CP_results_base2}].
The performances of the base model (Section \ref{sec:CP_base}) are compared with:
\begin{itemize}
  \item \textit{base search}: base model with search strategy described in Section \ref{sec:CP_search}
  \item \textit{base symmetry}: base model with symmetry breaking constraints described in \ref{sec:CP_symmetry} (no \textit{virtual} circuits)
  \item \textit{base search symmetry}: \textit{base symmetry} model with search strategy
\end{itemize}

\begin{figure}[H]
  \centering
  \includegraphics[width=1\textwidth]{03/results/base1.png}
  \caption{
    CP \(Base\) Model: comparison of time performance of \textit{base}, \textit{base search}, \textit{base symmetry} and \textit{base search symmetry} models
    for the first 20 instances.
  }
  \label{fig:CP_results_base1}
\end{figure}
\begin{figure}[H]
  \centering
  \includegraphics[width=1\textwidth]{03/results/base2.png}
  \caption{
    CP \(Base\) Model: comparison of time performance of \textit{base}, \textit{base search}, \textit{base symmetry} and \textit{base search symmetry} models
    for the first 20 instances.
  }
  \label{fig:CP_results_base2}
\end{figure}

The results are taken fixing the random seed, but should be taken in consideration that changing seed
the solve time of the same model on the same instance may vary a lot. As consequence we will describe 
only the most evident trends.\\

The first 20 instances do not show any particular difference in performances between the models.
We can already notice an increasing behaviour of solve time, but with some peaks for "random" instances showing 
how the solve time does not depend only on the number of circuits to place. \\

The secon 20 instances are not solved by all models, showing the first differences: the model solving 
the most of the instances is \textit{base search} and if we compare \textit{base symmetry} with 
\textit{base symmetry search} we have the proof that the search and restart strategies are quite effective.
We can also notice that most of the time adding the symmetry breaking constraints that we defined in 
Section \ref{sec:CP_symmetry} make the performances worse, confirming what already said.\\

The results obtained for the \(rotation\) models are plotted in the Figures [\ref{fig:CP_results_rotation1},\ref{fig:CP_results_rotation2}].
The performances of the rotation model (Section \ref{sec:CP_rotation}) are compared with:
\begin{itemize}
  \item \textit{rotation search}: rotation model with search strategy described in Section \ref{sec:CP_search}
  \item \textit{rotation symmetry}: rotation model with symmetry breaking constraints described in \ref{sec:CP_symmetry} (no \textit{virtual} circuits)
  \item \textit{rotation search symmetry}: \textit{rotation symmetry} model with search strategy
\end{itemize}

\begin{figure}[H]
  \centering
  \includegraphics[width=1\textwidth]{03/results/rotation1.png}
  \caption{
    CP \(rotation\) Model: comparison of time performance of \textit{rotation}, \textit{rotation search}, \textit{rotation symmetry} and \textit{rotation search symmetry} models
    for the first 20 instances.
  }
  \label{fig:CP_results_rotation1}
\end{figure}
\begin{figure}[H]
  \centering
  \includegraphics[width=1\textwidth]{03/results/rotation2.png}
  \caption{
    CP \(rotation\) Model: comparison of time performance of \textit{rotation}, \textit{rotation search}, \textit{rotation symmetry} and \textit{rotation search symmetry} models
    for the last 20 instances.
  }
  \label{fig:CP_results_rotation2}
\end{figure}

The results obtained for \textit{rotation} models are quite different from before: we still have an increasing solve 
time with some exceptional instances, but the first main difference is the impact of search strategies. 
While, with \textit{base} models, search strategies improved the efficiency, with \textit{rotation} model is the opposite.
We were not expecting any significant impact from the search strategy for the boolean variables \(r_i\) since their domain 
is highly limited and random search was not even implemented yet.
After checking the solutions and seeing that most of the \textit{makespan} values (in \textit{base} models) match with our 
definition of \textit{min\_makespan}, we may hazard that the dimensions of the circuits have been designed specifically for 
the \textit{VLSI} problem without rotations.\\

The second main difference from \textit{base} models is that in this case symmetry breaking constraint are more competitive. 
Maybe \textit{base} model was too simple and adding more constraints (referring to the symmetry breaking constraints) to 
compute was not balanced by the number of truncated solutions during the search.