\documentclass[a4paper, 11pt]{article}
\usepackage[top=3cm, bottom=3cm, left = 2cm, right = 2cm]{geometry} 
\geometry{a4paper} 
\usepackage[utf8]{inputenc}
\usepackage{textcomp}
\usepackage{graphicx} 
\usepackage{amsmath,amssymb}  
\usepackage{bm}  
\usepackage[pdftex,bookmarks,colorlinks,breaklinks]{hyperref}  
\hypersetup{linkcolor=blue,citecolor=blue,filecolor=blue,urlcolor=blue} % black links, for printed output
\usepackage{memhfixc} 
\usepackage{pdfsync}  
\usepackage{fancyhdr}
\pagestyle{fancy}
\usepackage[dvipsnames]{xcolor}
% \usepackage{babel}
\usepackage{amsmath}
\usepackage{subcaption}
\usepackage{float}

\usepackage{biblatex}
\addbibresource{bibliography.bib}

\graphicspath{{./assets/images/}}
\setlength{\parindent}{0pt}

\DeclareMathOperator*{\argmax}{argmax}
\DeclareMathOperator*{\argmin}{argmin}

%%%%%%%%%%%%%%%%%%%%%%%%%%%%%%%%%%%%%%%%%%%%%%%%%%%%%%%%%%%%%%%%%%%%%%%%%%%%%%%%%%%%%%%%%%%%%%%%%%%

\title{\huge{\textbf{Very Large Scale Integration Project}}}
\author{
    \begin{tabular}[t]{c@{\extracolsep{8em}}c}
                                                         &                                                    \\
        Davide Sangiorgi                                 & Leonardo Monti                                     \\
        \footnotesize{University of Bologna}             & \footnotesize{University of Bologna}               \\ 
        \small{\texttt{xxxxxx.xxxxx@studio.unibo.it}}    & \small{\texttt{leonardo.monti3@studio.unibo.it}}   \\
                                                         &                                                    \\
        Riccardo Falco                                   & Matteo Conti                                       \\
        \footnotesize{University of Bologna}             & \footnotesize{University of Bologna}               \\ 
        \small{\texttt{riccardo.falco2@studio.unibo.it}} & \small{\texttt{matteo.conti16@studio.unibo.it}}    \\
                                                         &                                 
    \end{tabular}   
}
\date{}

%%%%%%%%%%%%%%%%%%%%%%%%%%%%%%%%%%%%%%%%%%%%%%%%%%%%%%%%%%%%%%%%%%%%%%%%%%%%%%%%%%%%%%%%%%%%%%%%%%%

\begin{document}

\maketitle
\tableofcontents

%%%%%%%%%%%%%%%%%%%%%%%%%%%%%%%%%%%%%%%%%%%%%%%%%%%%%%%%%%%%%%%%%%%%%%%%%%%%%%%%%%%%%%%%%%%%%%%%%%%

\section{Introduction}\label{chapter:introduction}
    
\subsection{Background}
    \textit{Satisfiability Modulo Theories (SMT)} is the problem of determining whether a
    mathematical formula is satisfiable. It generalizes the Boolean satisfiability problem (SAT) to
    more complex formulas involving real numbers, integers, and/or various data structures such as 
    lists, arrays, bit vectors, and strings.

% % % % % % % % % % % % % % % % % % % % % % % % % % % % % % % % % % % % % % % % % % % % % % % % % %

\subsection{Formulation}
    \colorbox{BurntOrange}{TODO missing ...} \\

    In order to write a proper SMT formulation, we started from the SAT formulation (available in
    the Section \ref{chapter:SAT}) and we have replaced each boolean representation of integer 
    operators with the operators themself.\\

    In particular, the following replacements have been done:
    \begin{align*}
      \text{SMT} \ &\ \leftarrow\ \text{SAT}      \\
                 \cline{1-2}
        a \geq 1 \ &\ \leftarrow\ \text{at\_least\_one}(a) \\
        a \leq 1 \ &\ \leftarrow\ \text{at\_most\_one}(a)  \\
           a = 1 \ &\ \leftarrow\ \text{exactly\_one}(a)   \\
           a = b \ &\ \leftarrow\ \text{equal}(a,b)        \\
           a < b \ &\ \leftarrow\ lt(a,b)                  \\
        a \leq b \ &\ \leftarrow\ lte(a,b)                 \\
           a > b \ &\ \leftarrow\ gt(a,b)                  \\
        a \geq b \ &\ \leftarrow\ gte(a,b)                 \\
           a + b \ &\ \leftarrow\ sum\_b(a,b)              \\   
           a - b \ &\ \leftarrow\ sub\_b(a,b)                 
    \end{align*}

    The final formulation obtained is very "near" to the one defined for CP in the
    Section \ref{chapter:CP}.

% % % % % % % % % % % % % % % % % % % % % % % % % % % % % % % % % % % % % % % % % % % % % % % % % %

\subsection{Results}
    \colorbox{BurntOrange}{TODO missing ...} \\

    \newpage

\section{Background}\label{chapter:background}
    
\subsection{Background}
    \textit{Satisfiability Modulo Theories (SMT)} is the problem of determining whether a
    mathematical formula is satisfiable. It generalizes the Boolean satisfiability problem (SAT) to
    more complex formulas involving real numbers, integers, and/or various data structures such as 
    lists, arrays, bit vectors, and strings.

% % % % % % % % % % % % % % % % % % % % % % % % % % % % % % % % % % % % % % % % % % % % % % % % % %

\subsection{Formulation}
    \colorbox{BurntOrange}{TODO missing ...} \\

    In order to write a proper SMT formulation, we started from the SAT formulation (available in
    the Section \ref{chapter:SAT}) and we have replaced each boolean representation of integer 
    operators with the operators themself.\\

    In particular, the following replacements have been done:
    \begin{align*}
      \text{SMT} \ &\ \leftarrow\ \text{SAT}      \\
                 \cline{1-2}
        a \geq 1 \ &\ \leftarrow\ \text{at\_least\_one}(a) \\
        a \leq 1 \ &\ \leftarrow\ \text{at\_most\_one}(a)  \\
           a = 1 \ &\ \leftarrow\ \text{exactly\_one}(a)   \\
           a = b \ &\ \leftarrow\ \text{equal}(a,b)        \\
           a < b \ &\ \leftarrow\ lt(a,b)                  \\
        a \leq b \ &\ \leftarrow\ lte(a,b)                 \\
           a > b \ &\ \leftarrow\ gt(a,b)                  \\
        a \geq b \ &\ \leftarrow\ gte(a,b)                 \\
           a + b \ &\ \leftarrow\ sum\_b(a,b)              \\   
           a - b \ &\ \leftarrow\ sub\_b(a,b)                 
    \end{align*}

    The final formulation obtained is very "near" to the one defined for CP in the
    Section \ref{chapter:CP}.

% % % % % % % % % % % % % % % % % % % % % % % % % % % % % % % % % % % % % % % % % % % % % % % % % %

\subsection{Results}
    \colorbox{BurntOrange}{TODO missing ...} \\

    \newpage

\section{CP}\label{chapter:CP}
    
\subsection{Background}
    \textit{Satisfiability Modulo Theories (SMT)} is the problem of determining whether a
    mathematical formula is satisfiable. It generalizes the Boolean satisfiability problem (SAT) to
    more complex formulas involving real numbers, integers, and/or various data structures such as 
    lists, arrays, bit vectors, and strings.

% % % % % % % % % % % % % % % % % % % % % % % % % % % % % % % % % % % % % % % % % % % % % % % % % %

\subsection{Formulation}
    \colorbox{BurntOrange}{TODO missing ...} \\

    In order to write a proper SMT formulation, we started from the SAT formulation (available in
    the Section \ref{chapter:SAT}) and we have replaced each boolean representation of integer 
    operators with the operators themself.\\

    In particular, the following replacements have been done:
    \begin{align*}
      \text{SMT} \ &\ \leftarrow\ \text{SAT}      \\
                 \cline{1-2}
        a \geq 1 \ &\ \leftarrow\ \text{at\_least\_one}(a) \\
        a \leq 1 \ &\ \leftarrow\ \text{at\_most\_one}(a)  \\
           a = 1 \ &\ \leftarrow\ \text{exactly\_one}(a)   \\
           a = b \ &\ \leftarrow\ \text{equal}(a,b)        \\
           a < b \ &\ \leftarrow\ lt(a,b)                  \\
        a \leq b \ &\ \leftarrow\ lte(a,b)                 \\
           a > b \ &\ \leftarrow\ gt(a,b)                  \\
        a \geq b \ &\ \leftarrow\ gte(a,b)                 \\
           a + b \ &\ \leftarrow\ sum\_b(a,b)              \\   
           a - b \ &\ \leftarrow\ sub\_b(a,b)                 
    \end{align*}

    The final formulation obtained is very "near" to the one defined for CP in the
    Section \ref{chapter:CP}.

% % % % % % % % % % % % % % % % % % % % % % % % % % % % % % % % % % % % % % % % % % % % % % % % % %

\subsection{Results}
    \colorbox{BurntOrange}{TODO missing ...} \\

    \newpage

\section{SAT}\label{chapter:SAT}
    
\subsection{Background}
    \textit{Satisfiability Modulo Theories (SMT)} is the problem of determining whether a
    mathematical formula is satisfiable. It generalizes the Boolean satisfiability problem (SAT) to
    more complex formulas involving real numbers, integers, and/or various data structures such as 
    lists, arrays, bit vectors, and strings.

% % % % % % % % % % % % % % % % % % % % % % % % % % % % % % % % % % % % % % % % % % % % % % % % % %

\subsection{Formulation}
    \colorbox{BurntOrange}{TODO missing ...} \\

    In order to write a proper SMT formulation, we started from the SAT formulation (available in
    the Section \ref{chapter:SAT}) and we have replaced each boolean representation of integer 
    operators with the operators themself.\\

    In particular, the following replacements have been done:
    \begin{align*}
      \text{SMT} \ &\ \leftarrow\ \text{SAT}      \\
                 \cline{1-2}
        a \geq 1 \ &\ \leftarrow\ \text{at\_least\_one}(a) \\
        a \leq 1 \ &\ \leftarrow\ \text{at\_most\_one}(a)  \\
           a = 1 \ &\ \leftarrow\ \text{exactly\_one}(a)   \\
           a = b \ &\ \leftarrow\ \text{equal}(a,b)        \\
           a < b \ &\ \leftarrow\ lt(a,b)                  \\
        a \leq b \ &\ \leftarrow\ lte(a,b)                 \\
           a > b \ &\ \leftarrow\ gt(a,b)                  \\
        a \geq b \ &\ \leftarrow\ gte(a,b)                 \\
           a + b \ &\ \leftarrow\ sum\_b(a,b)              \\   
           a - b \ &\ \leftarrow\ sub\_b(a,b)                 
    \end{align*}

    The final formulation obtained is very "near" to the one defined for CP in the
    Section \ref{chapter:CP}.

% % % % % % % % % % % % % % % % % % % % % % % % % % % % % % % % % % % % % % % % % % % % % % % % % %

\subsection{Results}
    \colorbox{BurntOrange}{TODO missing ...} \\

    \newpage

\section{SMT}\label{chapter:SMT}
    
\subsection{Background}
    \textit{Satisfiability Modulo Theories (SMT)} is the problem of determining whether a
    mathematical formula is satisfiable. It generalizes the Boolean satisfiability problem (SAT) to
    more complex formulas involving real numbers, integers, and/or various data structures such as 
    lists, arrays, bit vectors, and strings.

% % % % % % % % % % % % % % % % % % % % % % % % % % % % % % % % % % % % % % % % % % % % % % % % % %

\subsection{Formulation}
    \colorbox{BurntOrange}{TODO missing ...} \\

    In order to write a proper SMT formulation, we started from the SAT formulation (available in
    the Section \ref{chapter:SAT}) and we have replaced each boolean representation of integer 
    operators with the operators themself.\\

    In particular, the following replacements have been done:
    \begin{align*}
      \text{SMT} \ &\ \leftarrow\ \text{SAT}      \\
                 \cline{1-2}
        a \geq 1 \ &\ \leftarrow\ \text{at\_least\_one}(a) \\
        a \leq 1 \ &\ \leftarrow\ \text{at\_most\_one}(a)  \\
           a = 1 \ &\ \leftarrow\ \text{exactly\_one}(a)   \\
           a = b \ &\ \leftarrow\ \text{equal}(a,b)        \\
           a < b \ &\ \leftarrow\ lt(a,b)                  \\
        a \leq b \ &\ \leftarrow\ lte(a,b)                 \\
           a > b \ &\ \leftarrow\ gt(a,b)                  \\
        a \geq b \ &\ \leftarrow\ gte(a,b)                 \\
           a + b \ &\ \leftarrow\ sum\_b(a,b)              \\   
           a - b \ &\ \leftarrow\ sub\_b(a,b)                 
    \end{align*}

    The final formulation obtained is very "near" to the one defined for CP in the
    Section \ref{chapter:CP}.

% % % % % % % % % % % % % % % % % % % % % % % % % % % % % % % % % % % % % % % % % % % % % % % % % %

\subsection{Results}
    \colorbox{BurntOrange}{TODO missing ...} \\

    \newpage


\section{ILP}\label{chapter:ILP}
    
\subsection{Background}
    \textit{Satisfiability Modulo Theories (SMT)} is the problem of determining whether a
    mathematical formula is satisfiable. It generalizes the Boolean satisfiability problem (SAT) to
    more complex formulas involving real numbers, integers, and/or various data structures such as 
    lists, arrays, bit vectors, and strings.

% % % % % % % % % % % % % % % % % % % % % % % % % % % % % % % % % % % % % % % % % % % % % % % % % %

\subsection{Formulation}
    \colorbox{BurntOrange}{TODO missing ...} \\

    In order to write a proper SMT formulation, we started from the SAT formulation (available in
    the Section \ref{chapter:SAT}) and we have replaced each boolean representation of integer 
    operators with the operators themself.\\

    In particular, the following replacements have been done:
    \begin{align*}
      \text{SMT} \ &\ \leftarrow\ \text{SAT}      \\
                 \cline{1-2}
        a \geq 1 \ &\ \leftarrow\ \text{at\_least\_one}(a) \\
        a \leq 1 \ &\ \leftarrow\ \text{at\_most\_one}(a)  \\
           a = 1 \ &\ \leftarrow\ \text{exactly\_one}(a)   \\
           a = b \ &\ \leftarrow\ \text{equal}(a,b)        \\
           a < b \ &\ \leftarrow\ lt(a,b)                  \\
        a \leq b \ &\ \leftarrow\ lte(a,b)                 \\
           a > b \ &\ \leftarrow\ gt(a,b)                  \\
        a \geq b \ &\ \leftarrow\ gte(a,b)                 \\
           a + b \ &\ \leftarrow\ sum\_b(a,b)              \\   
           a - b \ &\ \leftarrow\ sub\_b(a,b)                 
    \end{align*}

    The final formulation obtained is very "near" to the one defined for CP in the
    Section \ref{chapter:CP}.

% % % % % % % % % % % % % % % % % % % % % % % % % % % % % % % % % % % % % % % % % % % % % % % % % %

\subsection{Results}
    \colorbox{BurntOrange}{TODO missing ...} \\

    \newpage

% \bibliographystyle{abbrv}
% \bibliography{bibliography}
\nocite{*}
\printbibliography

%%%%%%%%%%%%%%%%%%%%%%%%%%%%%%%%%%%%%%%%%%%%%%%%%%%%%%%%%%%%%%%%%%%%%%%%%%%%%%%%%%%%%%%%%%%%%%%%%%%

\end{document}
